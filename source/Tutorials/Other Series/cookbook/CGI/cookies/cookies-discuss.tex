
\subsubsection{Idea}

Cookies are a mechanism by which the {\tt HTTP} protocol, which is
stateless, can add persistence to a session. This is initiated by the
server, which sends a cookie that the client stores locally. The
client subsequently includes the cookie in its HTTP headers when it
requests a page from the server (precisely what servers will receive a
cookie is controlled by the path and domain properties of the cookie;
cookies are sent with requests whose URL matches the path and the
domain). The server could use cookies to store information that it has
received from the client through a form submission. Usually just one
cookie is used, however, namely the session id. The server maps the
session id to the session state data.

We add a category to {\tt NSMutableDictionary} that contains a method
that initializes the dictionary with the variables that were passed in
through the environment variable �{\tt HTTP\_COOKIE.} We also implement
a method that outputs the content of a dictionary as a series of HTTP
{\tt Set-Cookie} headers.

Our test application adds two cookies to the set of cookies it has
received from the client. These cookies have a lifetime of thirty
seconds. The value of each cookie is the time since the Epoch
(00:00:00 UTC, January 1, 1970), measured in seconds, so that the
server can output the remaining time for each cookie it receives.

The syntax of the {\tt Set-Cookie} header is like this (all of it on
one line):
\begin{verbatim}
            Set-Cookie: KEY=VALUE
            [; expires=DATE] [; path=PATH]
            [; domain=DOMAIN_NAME] [; secure]
\end{verbatim}

The date is required to have the format {\tt Wdy, DD-Mon-YYYY HH:MM:SS
GMT.} 

The {\tt Cookie} header that is returned by the client looks like
this: 
\begin{verbatim}
            KEY1=VALUE1; KEY2=VALUE2; ...
\end{verbatim}

Note that the protocol as we use it here lets the server receive
multiple key-value pairs in a single {\tt Cookie} header, whereas we
must send one {\tt Set-Cookie} header for each pair we wish to set.

\subsubsection{Preliminaries}

We use a shell script as a wrapper; it should go into the CGI-BIN
directory of your webserver, e.g. you could save it as {\tt
/cgi-bin/cookies.sh.}

\begin{verbatim}
#! /bin/sh
#

export GNUSTEP_SYSTEM_ROOT=/usr/GNUstep
export TZ=Europe/Berlin
. $GNUSTEP_SYSTEM_ROOT/System/Makefiles/GNUstep.sh

/home/gnustep/cookies/shared_obj/ix86/linux-gnu/gnu-gnu-gnu/cookies
\end{verbatim} %%$

\subsubsection{Implementation}

Start by declaring {\tt PERSIST}, a constant that defines the
persistence of the cookies we produce. The method {\tt initWithCookie}
parses {\tt HTTP\_COOKIE} and stores the key-value pairs in the
dictionary. The method {\tt cookieToString} returns a string that is
suitable for inclusion among the headers sent to the client. The
special keys {\tt expires, domain, path} and {\tt secure} can be used
to set the properties of the cookie.

\begin{lstlisting}{}
#include <Foundation/Foundation.h>

#define PERSIST 30

@interface NSMutableDictionary (Cookies)

- (NSMutableDictionary *)initWithCookie;
- (NSString *)cookieToString;

@end
\end{lstlisting}

The first step is to retrieve the cookie from the process
environment. There is work to be done if {\tt HTTP\_COOKIE} was not
empty.

\begin{lstlisting}{}
@implementation NSMutableDictionary (Cookies)

- (id)initWithCookie
{
    NSString *cookie = 
        [[[NSProcessInfo processInfo] environment]
            objectForKey:@"HTTP_COOKIE"];

    [self initWithCapacity:1];

    if(cookie!=nil){
\end{lstlisting}

The algorithm works works like this: scan up to the first semicolon
and split the key-value pair on the equal sign. Skip whatever
whitespace you may find. Repeat until you reach the end of the string.

We need a character set for whitespace so that we may skip those
characters. We intialize the scanner with the cookie and start a loop
that ends when the scanner reaches the end of the string.

\begin{lstlisting}{}
        NSCharacterSet *space =
            [NSCharacterSet whitespaceAndNewlineCharacterSet];
        NSScanner *scn = 
            [NSScanner scannerWithString:cookie];

        while([scn isAtEnd]==NO){
\end{lstlisting}

We scan up to the first semicolon; {\tt pair} holds those characters,
but does not include the semicolon. We split the pair on the equal
sign.

\begin{lstlisting}{}
            NSString *pair;
            if([scn scanUpToString:@";" intoString:&pair]==YES){
                NSArray *entry = 
                    [pair componentsSeparatedByString:@"="];
                NSString *key, *value;
\end{lstlisting}

We do some error checking. There should be exactly two fields: a key
and a value. We raise an exception if this is not the case.

\begin{lstlisting}{}
                if([entry count]!=2){
                    [NSException 
                        raise:NSRangeException 
                        format:@"cookie \"%@\""
                        @"contains bad key-value pair; "
                        @"%d fields, should be 2; ",
                        cookie, [entry count]];
                }
\end{lstlisting}

Neither the key nor the value may be empty.

\begin{lstlisting}{}
                key = [entry objectAtIndex:0];
                value = [entry objectAtIndex:1];

                if(![key length] || ![value length]){
                    [NSException 
                        raise:NSRangeException 
                        format:@"cookie \"%@\":"
                        @"key and value must not be empty;"
                        @"got lengths %d (key) and %d (value).", 
                        cookie, [key length], [value length]];
                }
\end{lstlisting}

If we pass the two checks then we have found a good key-value pair and
add it to the dictionary. We skip over the semicolon and any
whitespace. This takes us to the first character of the next pair or
to the end of the string.

\begin{lstlisting}{}
                [self setObject:value forKey:key];

                [scn scanString:@";" intoString:NULL];
                [scn scanCharactersFromSet:space  
                     intoString:NULL];
            }
        }
    }
\end{lstlisting}

We return the dictionary when we are done scanning.

\begin{lstlisting}{}
    return self;
}
\end{lstlisting}

The method {\tt cookieToString} builds a set of {\tt Set-Cookie}
headers from the contents of the dictionary. It takes special care to
process the cookie properties properly. There are two steps: first,
processs all keys in the dictionary to build the cookie's properties
as well as an array of key-value assignments and second, output a {\tt
Set-Cookie} header for each pair, using the properties from the first
step.

The array {\tt data} holds key-value pairs and the array {\tt
properties} holds properties. There is a boolean to indicate whether
the cookie is marked secure or not. The variable {\tt field} holds a
single item being added to one of the arrays. We iterate over all keys
of the dictionary.

\begin{lstlisting}{}
- (NSString *)cookieToString
{
    NSMutableArray *data = 
        [NSMutableArray arrayWithCapacity:1];
    NSMutableArray *properties =
        [NSMutableArray arrayWithCapacity:1];
    BOOL secure = NO;
    
    NSString *field;

    NSEnumerator *keyEnum = [self keyEnumerator];
    NSString *key;
    while((key=[keyEnum nextObject])!=nil){
\end{lstlisting}

The properties {\tt domain} and {\tt path} are easy: we build the
key-value pair and add it to the array of properties.

\begin{lstlisting}{}
        if([key isEqualToString:@"domain"]==YES ||
           [key isEqualToString:@"path"]==YES){
            field = 
                [NSString stringWithFormat:@"%@=%@;",
                          key, [self objectForKey:key]];
            [properties addObject:field];
        }
\end{lstlisting}

The property {\tt expires} is assumed to be a date object. We require
a date string in the format that the protocol uses. We use a calendar
format to obtain the result; note that we must use {\tt GMT} for the
time zone.

\begin{lstlisting}{}
        else if([key isEqualToString:@"expires"]==YES){
           NSString *dateStr =
               [[self objectForKey:key] 
                   descriptionWithCalendarFormat:@"%a, %d-%b-%Y "
                   @"%H:%M:%S GMT" 
                   timeZone:[NSTimeZone timeZoneWithName:@"GMT"]
                   locale:nil];
            field = 
                [NSString stringWithFormat:@"%@=%@;",
                          key, dateStr];
            [properties addObject:field];
        }
\end{lstlisting}

We set the boolean indicator for the property {\tt secure} if it is
among the keys in the dictionary. All other entries of the dictionary
are assumed to be data, i.e. key-value pairs. We format each pair
according to the requirements and store it in the array {\tt
data}. This completes the first step.

\begin{lstlisting}{}
        else if([key isEqualToString:@"secure"]==YES){
            secure = YES;
        }
        else{
            field =  
                [NSString stringWithFormat:@"%@=%@;",
                          key, [self objectForKey:key]];
            [data addObject:field];
        }
    }
\end{lstlisting}

We now have all the properties. This means that we can construct the
property string, which stays the same for all pairs. We build a
string that contains all properties separated by single spaces. If
the property {\tt secure} has been set, than it is included at the
end of the property string.

\begin{lstlisting}{}
    NSString *props = 
        [properties componentsJoinedByString:@" "];
    if(secure==YES){
        props = [props stringByAppendingString:@" secure"];
    }
\end{lstlisting}

The second step is to build the result string, which starts out
empty. We iterate over all data lines and construct a {\tt Set-Cookie}
header by including first the data, and then the properties. We return
the result when we are done.

\begin{lstlisting}{}
    NSString *result = @"", *line;
    NSEnumerator *lineEnum = [data objectEnumerator];
    while((line=[lineEnum nextObject])!=nil){
        NSString *allFields =
            [NSString 
                stringWithFormat:@"Set-Cookie: %@ %@\r\n", 
                line, props];
        result = 
            [result stringByAppendingString:allFields];
    }

    return result;
}
\end{lstlisting}

The remainder of this section shows how we test the code that we
discussed above. Recall that our test program should add two new
cookies each time it is invoked and display the cookies that it
received and the time that remains until they expire. 

We begin by obtaining the cookies that may have been passed in
through {\tt HTTP\_COOKIE}. We also obtain a date that is thirty
seconds in the future, or some other defined value. It determines the
expiry of the new cookies that we are about to set.

\begin{lstlisting}{}
#define NEWKEYS 2

int main(int argc, char** argv, char **env)
{
    NSAutoreleasePool *pool = [NSAutoreleasePool new];

    NSMutableDictionary *cookieValues =
        [[NSMutableDictionary alloc] initWithCookie];

    NSDate *date = 
        [NSDate dateWithTimeIntervalSinceNow:PERSIST];
\end{lstlisting}

The new cookie set starts out containing two properties, the date and
the path.

\begin{lstlisting}{}
    NSMutableDictionary *newCookie = 
        [NSMutableDictionary 
            dictionaryWithObjectsAndKeys:date, @"expires",
            @"/cgi-bin/cookies.sh", @"path", nil];
\end{lstlisting}

We must now add the key-value pairs, {\tt NEWKEYS} items to be
precise. The key should be a string containing the character 'k', the
last four digits of the time, and a three digit index to make it
unique, and its value, most importantly, should be the current time,
i.e. the time when it was created.

\begin{lstlisting}{}
    long now = time(NULL); int k;
    srand48(now);

    for(k=0; k<NEWKEYS; k++){
        NSString 
            *key = 
            [NSString stringWithFormat:@"k%04ld%03d",
                      now%10000, k],
            *value = 
            [NSString stringWithFormat:@"%ld",
                      now];
        [newCookie setObject:value forKey:key];
    }
\end{lstlisting}

We are now ready to output HTML code. There are two sections: a list
of all cookies that we received and how much longer each will last,
some auxiliary information and a button that invokes {\tt cookies.sh.}
The set of cookies comes first, followed by the header and the
beginning of the body of the document.

\begin{lstlisting}{}
    printf("%s", [[newCookie cookieToString] cString]);
    printf("Content-type: text/html\r\n\r\n");

    printf("<HTML><HEAD><TITLE>Cookies</TITLE></HEAD><BODY>\n");
\end{lstlisting}

We prepare to iterate over the key-value pairs that we received. There
is an enumerator and a counter. The data will not use HTML markup, so
we output a PRE tag.

\begin{lstlisting}{}
    NSEnumerator *en;
    NSString *key, *value;

    int count = 1;
    long rem;

    puts("<PRE>");
\end{lstlisting}

First we print the data (key-value pairs) that we received; this helps
to debug the program. We sort the keys in lexical order and prepare to
iterate over the array thus obtained.

\begin{lstlisting}{}
    printf("Received:\n%s\n\n\n", 
           [[cookieValues cookieToString] cString]);

    en = [[[cookieValues allKeys]
              sortedArrayUsingSelector:@selector(compare:)]
             objectEnumerator];
    while((key = [en nextObject])!=nil){
\end{lstlisting}

We lookup up the value associated to each key. The remaining time for
this key is the difference between the time that has elapsed since its
creation and the total persistence time. Note that differences between
the clocks of the server and the client can cause the client to remove
the cookie a bit sooner or a bit later than required. We construct a
string that contains the key, the value, and the remaining time, print
it and increment the counter.

\begin{lstlisting}{}
        value =  [cookieValues objectForKey:key];
	rem = (long)PERSIST-(now-atol([value cString]));

        NSString *item =
	    [NSString stringWithFormat:@"key %d: %@\n"
                      @"value %d: %@\n"
                      @"remaining %d: %lds\n",
		      count, key, count, value,
                      count, rem];

	printf("%s\n", [item cString]);
	count++;
    }
\end{lstlisting}

We output an indicator phrase that shows whether a cookie was received. We
also print the {\tt Set-}{\tt Cookie} headers that we sent earlier, so
the user knows what cookies to expect when he submits the form.

\begin{lstlisting}{}

    printf("%s receive a cookie.\n\n\n",
           ([cookieValues count]? "Did" : "Did not"));
    printf("%s\n", [[newCookie cookieToString] cString]);
\end{lstlisting}

The second section of the document contains a button that reloads the
script, thereby forcing the client to send the current set of cookies
to the server.

\begin{lstlisting}{}
    puts("</PRE>");

    printf("<FORM ACTION=cookies.sh METHOD=GET>\n"
           "<INPUT TYPE=SUBMIT VALUE=Submit>\n"
           "</FORM>\n");

    printf("</BODY></HTML>\n");

    [pool release];
    exit(0);
}
\end{lstlisting}

