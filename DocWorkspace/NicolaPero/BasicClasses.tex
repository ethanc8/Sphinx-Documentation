%%%
%%% Nicola Pero
%%%
%%%
%%%
%%%
%%%

\documentclass[a4paper]{article}
\begin{document}
\author{Nicola Pero nicola.pero@meta-innovation.com}
\title{Basic GNUstep Base Library Classes}
\date{First Edition: June 2000; Last Updated: January 2008}
\maketitle

\section{NSString}
\subsection{What is it}
Instances of the \texttt{NSString} class represent strings composed of
unicode characters.  Unicode characters are described by a 16-bit
integer; this type is called \texttt{unichar} in GNUstep, but you will
very rarely (if ever) need to work with it directly.  Unicode is a
standard which supports the character sets of nearly all human
languages.

\subsection{Static Instances}
The easiest way to create a \texttt{NSString} is by creating a `static' 
instance of it using the \texttt{@"..."} construct.  For example, 
\begin{verbatim}
NSString *name = @"Nicola";
\end{verbatim}
\texttt{name} will point then to a `static' instance of
\texttt{NSString} containing a unicode representation of the ASCII
string \texttt{Nicola}.  A `static' instance basically means an
instance which is allocated at compile time, whose instance variables
are fixed, and which can never be released.

\subsection{stringWithFormat}
The other main way of creating \texttt{NSString}s is by means of the
class method \texttt{+stringWithFormat:}.  This is a method of the
class \texttt{NSString}; it accepts a list of arguments which are
processed in a way very similar to how the \texttt{printf} function of
the standard C library processes its arguments; the difference is that
the result of the method is not sent to the standard output, but
rather it is put in a \texttt{NSString} which is then returned as
return value of the method.

Here is an example: 
\begin{verbatim}
int age = 25;
NSString *message;

message = [NSString stringWithFormat: @"Your age is %d", age];
\end{verbatim}

\texttt{message} will be a \texttt{NSString} containing "Your age is 25".

A special feature of \texttt{stringWithFormat} is that it recognises
the \texttt{\%@} conversion specification.  You can use this to 
specify another \texttt{NSString} (in the same way as you would use 
\texttt{\%s} to specify another C string): 
\begin{verbatim}
NSString *first;
NSString *second;

first = @"Nicola";
second = [NSString stringWithFormat: @"My name is %@", first];
\end{verbatim}
this code will cause the \texttt{second} variable to be set to 
the string \texttt{My name is Nicola}. 

More generally, you can use the \texttt{\%@} specification to output a
description of an object (as returned by the \texttt{NSObject}'s
\texttt{-description}).  This is often useful in debugging, as in: 
\begin{verbatim}
NSObject *obj = [anObject someMethod];

NSLog (@"The method returned: %@", obj);
\end{verbatim}

\subsection{Converting to and from C strings}
It is often useful to be able to create a \texttt{NSString} from a
standard ASCII (or more generally UTF-8) C string (not fixed at
compile time).  Say for example that our program needs to call a C
library function
\begin{verbatim}
char *function (void);
\end{verbatim}
which returns some useful information in a C string.  We want to
create a \texttt{NSString} using the contents of the C string. 
The simplest way to do it is by using the \texttt{NSString}'s 
class method \texttt{+stringWithUTF8String:}, as follows: 
\begin{verbatim}
char *result;
NSString *string;

result = function ();
string = [NSString stringWithUTF8String: result];
\end{verbatim}
Sometimes we need to do the reverse, i.e. to convert a
\texttt{NSString} to a standard C ASCII (or more generally UTF-8)
string.  We can do it using the \texttt{-UTF8String} method of the
\texttt{NSString} class, as in the following example:

\begin{verbatim}
char *result;
NSString *string;

string = @"Hello";
result = [string UTF8String];
\end{verbatim}

\subsection{NSMutableString}
\texttt{NSString}s are immutable objects, that is, once you create an 
\texttt{NSString}, you can not modify it.  This allows the GNUstep 
libraries to optimise the \texttt{NSString} code.  
If you need to be able to modify a string, you should use a special
subclass of \texttt{NSString}, called \texttt{NSMutableString}.
Since \texttt{NSMutableString} is a subclass of \texttt{NSString}, 
you can use a \texttt{NSMutableString} wherever a \texttt{NSString} 
could be used.  But, a \texttt{NSMutableString} responds to methods 
which allow you to modify the string directly, a thing you can't 
do with a generic \texttt{NSString}. 

To create a \texttt{NSMutableString}, you can use \texttt{+stringWithFormat:}, 
as in the following example: 
\begin{verbatim}
NSString *name = @"Nicola";
NSMutableString *str;

str = [NSMutableString stringWithFormat: @"Hello, %@", name];
\end{verbatim}
While \texttt{NSString}'s implementation of \texttt{+stringWithFormat:}
returns a \texttt{NSString}, \texttt{NSMutableString}'s implementation 
returns a \texttt{NSMutableString}.  Static strings created with the  
\texttt{@"..."} construct are always immutable.

In practice, \texttt{NSMutableString}s are not used very often,
because usually if you want to modify a string you just create a new
string derived from the one you already have.  

The most interesting method of the \texttt{NSMutableString} class is
possibly the method \texttt{-appendString:}.  It takes as argument a
\texttt{NSString}, and appends it to the receiver.

For example, the following code: 
\begin{verbatim}
NSString *name = @"Nicola";
NSString *greeting = @"Hello";
NSMutableString *s;

s = AUTORELEASE ([NSMutableString new]);
[s appendString: greeting];
[s appendString: @", "];
[s appendString: name];
\end{verbatim}
(where we used \texttt{new} to create a new empty
\texttt{NSMutableString}) produces the same result as the following
one:
\begin{verbatim}
NSString *name = @"Nicola";
NSString *greeting = @"Hello";
NSMutableString *s;

s = [NSMutableString stringWithFormat: @"%@, %@", 
                                       greeting, name];
\end{verbatim}

\subsection{Reading and Saving Strings to/from Files}
We don't have the time to describe all the string-related features of
the GNUstep base library; but it's useful to have at least a quick
look at how easy is writing/reading strings to/from files.

If you need to write the contents of a string into a file, you can use
the method \texttt{-writeToFile:atomically:}, as shown in the
following example:
\begin{verbatim}
#include <Foundation/Foundation.h>

int
main (void)
{
  NSAutoreleasePool *pool = [NSAutoreleasePool new];
  NSString *name = @"This string was created by GNUstep";

  if ([name writeToFile: @"/home/nico/testing" atomically: YES])
    {
      NSLog (@"Success");
    }
  else 
    {
      NSLog (@"Failure");
    }
  return 0;
}
\end{verbatim}
\texttt{writeToFile:atomically} returns \texttt{YES} upon success, 
and \texttt{NO} upon failure.  If the atomically flag is \texttt{YES}, 
then the library first writes the string into a file with a temporary 
name, and, when the writing has been successfully done, renames the 
file to the specified filename.  This prevents erasing the previous 
version of filename unless writing has been successful.  Usually, this 
is a very good feature, which you want to enable.

Reading the contents of a file into a string is easy too.  You can
simply use \texttt{+stringWithContentsOfFile:}, as in the following
example, which reads \texttt{@"/home/nicola/test"}:
\begin{verbatim}
#include <Foundation/Foundation.h>

int
main (void)
{
  NSAutoreleasePool *pool = [NSAutoreleasePool new];
  NSString *string;
  NSString *filename = @"/home/nico/test";

  string = [NSString stringWithContentsOfFile: filename];
  if (string == nil)
    {
      NSLog (@"Problem reading file %@", filename);
      // <missing code: do something to manage the error...>
      // <exit perhaps ?>
    }

  /*
   * <missing code: do something with string...>
   */

  return 0;
}
\end{verbatim}

\section{NSArray}
\subsection{What is it}
Instances of \texttt{NSArray} represent arrays of objects. 
That is, a \texttt{NSArray} is used to store an ordered set 
of objects.

\subsection{Comparison with Pure C Arrays}
In a GNUstep tool or application, you can use pure C arrays as well,
exactly as you do in C.  \texttt{NSArray}s have some advantages and
disadvantages over pure C arrays.  The first advantage is that the
programmer interface of \texttt{NSArray} is slightly easier, which
makes your code simpler to read, maintain and debug.  In particular,
if you use arrays which can be modified (\texttt{NSMutableArray}s),
the GNUstep library shrinks or expands the array automatically for you
as needed when you add or remove objects, without you having to
manually allocate or resize the memory needed for the array.  The
second advantage of \texttt{NSArray}s is that they provide facilities
to do things which are not necessarily straightforward to do with pure
C arrays.  In the last section of this tutorial we will be learning
about one of this facilities, the ability of saving an array of
strings (and other simple objects) into a plain text file, and
automatically recreating the array by reading the information from the
file.  The main disadvantage is that a \texttt{NSArray} is slower than a
pure C array, but you should not overestimate this problem, which
becomes important only when you need really fast code and have to
iterate over really big arrays.  In most cases, using a C array or a
\texttt{NSArray} does not make any real difference on the performance;
but of course there are cases in which it does.

\subsection{NSArrays are immutable}
As in the case of strings, \texttt{NSArray}s are immutable; you can't 
change them after creation (this allows the GNUstep \texttt{NSArray} 
code to use some optimisations which would be impossible otherwise).  
If you need an array that you can modify, you can use a subclass
of \texttt{NSArray}, called \texttt{NSMutableArray}.  This happens 
quite often, so we'll discuss it in some details later on. 

\subsection{arrayWithObjects}
To create an \texttt{NSArray}, you usually use the class method
\texttt{+arrayWithObjects:}, which takes as arguments 
a \texttt{nil}-terminated list of objects to put in the array:
\begin{verbatim}
NSArray *names;

names = [NSArray arrayWithObjects: @"Nicola", @"Margherita",
                                   @"Luciano", @"Silvia", nil];
\end{verbatim}
The last element in the list must be \texttt{nil}, to mark the end of
the list.  You can put in the same array objects of different classes:
\begin{verbatim}
NSArray *objects;
NSButton *buttonOne;
NSButton *buttonTwo;
NSTextField *textField;

buttonOne = AUTORELEASE ([NSButton new]);
buttonTwo = AUTORELEASE ([NSButton new]);
textField = AUTORELEASE ([NSTextField new]);

objects = [NSArray arrayWithObjects: buttonOne, buttonTwo, 
                                     textField, nil];
\end{verbatim}
but you can't put \texttt{nil} inside an array.  All objects 
in an array must be valid, not-\texttt{nil} objects.

When an object is put in an array, the \texttt{NSArray} sends to it a
\texttt{-retain} message, to prevent it from being deallocated while
it is still in the array.  When an object is removed from the array
(because the array is a \texttt{NSMutableArray} so objects can be
removed from it, or because the array itself is deallocated), it
is sent a \texttt{-release} message.

\subsection{Accessing Objects in a NSArray}
To access an object in an \texttt{NSArray}, you use the 
\texttt{-objectAtIndex:} method, as in the following example: 
\begin{verbatim}
NSArray *numbers;
NSString *string;

numbers = [NSArray arrayWithObjects: @"One", @"Two", @"Three", 
                                     nil];
string = [numbers objectAtIndex: 2];   // @"Three"
\end{verbatim}
Of course, you have to be careful not to ask for an object at an index
which is negative or bigger than the size of the array; if you do, an
\texttt{NSRangeException} is raised (we'll learn more about exceptions
in another tutorial).

To get the length of an array, you use the method \texttt{-count}, 
as in: 
\begin{verbatim}
NSArray *numbers;
int i;

numbers = [NSArray arrayWithObjects: @"One", @"Two", @"Three", 
                                     nil];
i = [numbers count];   // 3
\end{verbatim}

\subsection{Describing an Array}
It is sometime useful for debugging to have a look at what is stored
inside an \texttt{NSArray}.  Nothing easier with the GNUstep base
library: all methods and functions taking format arguments (in the way
\texttt{+stringWithFormat:} does) recognise the conversion specification
\texttt{\%@}, which describes an object.  So, to describe the 
\texttt{NSArray} called \texttt{array}, you can just do:
\begin{verbatim}
NSLog (@"Array: %@", array);
\end{verbatim}

For example, an array created with 
\begin{verbatim}
NSArray *array;

array = [NSArray arrayWithObjects: @"Hi", @"Hello", 
                                   AUTORELEASE ([NSLock new]), 
                                   nil];
\end{verbatim}
will be described by the previous \texttt{NSLog} call as: 
\begin{verbatim}
Jun 08 08:50:38 Test[16808] Array: (Hi, Hello, <NSLock: 8081f98>)
\end{verbatim}

You may note that it is not as easy to get a full description of a
pure C array.

\subsection{Iterating over Array Elements}
\subsubsection{First Way - Using objectAtIndex}
To iterate over the elements of an array, you may simply use a C-like
approach, as in the following debugging routine which prints out the
description of all the elements in an \texttt{NSArray}:
\begin{verbatim}
void
describeArray (NSArray *array)
{
  int i, count;

  count = [array count];
  for (i = 0; i < count; i++)
  {
    NSLog (@"Object at index %d is: %@", 
           i, [array objectAtIndex: i]);
  }
}
\end{verbatim}
Of course, this code is in a certain sense useless, because you can
just get the complete description of the array in a single
\texttt{NSLog} call, as shown in the previous section; but it fulfils
its purpose, which is to show a concrete example of how to iterate on
array elements.

\subsubsection{Second Way - Using objectEnumerator}
There is another very important way to iterate over the elements of an
array, and it is by using the \texttt{-objectEnumerator} method.  This
method returns an object of class \texttt{NSEnumerator}, which can be
used to enumerate the objects in the array.  The only real thing you
need to know about \texttt{NSEnumerator} is that it has a method
called \texttt{-nextObject}.  The first time you invoke it, it returns
the first object in the array.  The second time you invoke it, it
returns the second element on the array, and so on till there are no
more objects in the array; at this point, the \texttt{NSEnumerator}
returns \texttt{nil}.  In the following example, the code to describe
an array is rewritten in this second way:
\begin{verbatim}
void
describeArray (NSArray *array)
{
  NSEnumerator *enumerator;
  NSObject *obj;

  enumerator = [array objectEnumerator];

  while ((obj = [enumerator nextObject]) != nil)
    {
      NSLog (@"Next Object is: %@", obj);      
    }
  }
}
\end{verbatim}
This second way is generally slightly faster than the first one but
has a very important restriction: you should not modify the array (if
it is a \texttt{NSMutableArray}) while enumerating its elements in
this way.  Be careful about this problem, because it is easy to forget
this condition - this would introduce subtle bugs.

\subsection{Searching for an Object}
If you want to check whether an \texttt{NSArray} contains a certain
object or not, you should use the \texttt{-containsObject:} method, 
as in the following example: 
\begin{verbatim}
NSArray *array;

array = [NSArray arrayWithObjects: @"Nicola", @"Margherita",
                                   @"Luciano", @"Silvia", nil];

if ([array containsObject: @"Nicola"]) // YES
  {
    // Do something
  }
\end{verbatim}
\texttt{-containsObject:} compares the objects using
\texttt{-isEqual:}, which is usually what you want: eg, two
\texttt{NSString} objects containing the same UNICODE characters would
be considered equal, even if they are not the same object.

To get the index of an object, you can use \texttt{-indexOfObject:},
which returns the index of the object (better, of an object equal to
the argument), or the constant \texttt{NSNotFound} if no object equal
to the argument can be found in the array, as in the following
example:
\begin{verbatim}
NSArray *array;
int i, j;

array = [NSArray arrayWithObjects: @"Nicola", @"Margherita",
                                   @"Luciano", @"Silvia", nil];

i = [array indexOfObject: @"Margherita"]) // 1
j = [array indexOfObject: @"Luca"]) // NSNotFound
\end{verbatim}

\subsection{NSMutableArray}
If you need to add, remove or replace objects in an array, then you
should use a \texttt{NSMutableArray}.  Generally, you create
\texttt{NSMutableArray}s by simply using 
\begin{verbatim}
NSMutableArray *array;

array = [NSMutableArray new];
\end{verbatim}
(you then need to \texttt{AUTORELEASE} it if needed).  
This creates an \texttt{NSMutableArray} containing no elements. 

\subsubsection{Adding an Object}
To add an element at the end of the array, you can use
\texttt{addObject}, as in:
\begin{verbatim}
NSMutableArray *array;

array = [NSMutableArray new];
[array addObject: anObject];
\end{verbatim}

Assuming \texttt{anObject} is an \texttt{NSObject} (but not
\texttt{nil}, remember, you can't put a \texttt{nil} object into an
\texttt{NSArray}).  As usual, \texttt{anObject} is \texttt{RETAIN}ed 
when it is added to the array.

If you want to insert an object into an array at a certain position, 
you can use \texttt{insertObject:atIndex:}:
\begin{verbatim}
NSMutableArray *array;

array = [NSMutableArray new];
[array addObject: @"Michele"];
[array addObject: @"Nicola"];
[array insertObject: @"Alessia" atIndex: 1];

/* Now the array contains Michele, Alessia, Nicola. */
\end{verbatim}

\subsubsection{Removing an Object}
To remove an object, you can use \texttt{removeObjectAtIndex:}, as in 
\begin{verbatim}
NSMutableArray *array;

array = [NSMutableArray new];
[array addObject: @"Michele"];
[array addObject: @"Nicola"];
[array insertObject: @"Alessia" atIndex: 1];

/* Now the array contains Michele, Alessia, Nicola. */

[array removeObjectAtIndex: 0];

/* Now the array contains Alessia, Nicola. */
\end{verbatim}
When an object is removed from the array, it receives a \texttt{release} 
message.  This balances the \texttt{retain} which was sent to the object 
when it was first added to the array, and allows the object to be 
deallocated, if needed.

\subsubsection{Replacing an Object}
To replace an object, you can use
\texttt{replaceObjectAtIndex:withObject:}, as in
\begin{verbatim}
NSMutableArray *array;

array = [NSMutableArray new];
[array addObject: @"Alessia"];
[array addObject: @"Michele"];

/* Now the array contains Alessia, Michele. */

[array replaceObjectAtIndex: 1  withObject: @"Nicola"];

/* Now the array contains Alessia, Nicola. */
\end{verbatim}
The object which is removed from the array (because it is being
replaced) receives a \texttt{release} message; the object which is
added to the array (because it replaces the other object) receives a
\texttt{retain} message.

\section{NSDictionary}
\subsection{What is it}
An instance of \texttt{NSDictionary} contains a set of \emph{keys}, and 
for each key, an associate \emph{value}.  Both keys and values must 
be objects; the keys must all be different from each other, and must 
not be \texttt{nil}; the values must not be \texttt{nil}.

An example of \texttt{NSDictionary} may be represented as follows: 
\begin{verbatim}
{
  Luca = "/opt/picture.png";
  "Birthday Photo" = "/home/nico/birthday.png";
  "Birthday Image" = "/home/nico/birthday.png";
  "My Sister" = "/home/marghe/pic.jpg";
}
\end{verbatim}
In this example, for each key (\texttt{Luca}, etc), which is a string
(an image name), there is associated a value
(\texttt{/opt/picture.png}), which is a string (the full path of the
file containing the image).  Note that different keys may have the
same values.  In this case, the same actual graphic file can be
accessed using two different names.

In this example, all the objects were strings, but that not need be
always the case; keys and values may be arbitrary objects (and not
necessarily of the same class).  A basic difference between
\texttt{NSArray} and \texttt{NSDictionary} is that the elements
contained in an \texttt{NSArray} are lined up in a precise order,
while the couples key/value contained in a \texttt{NSDictionary} are
\emph{not} ordered at all.  You usually access an object in an array
by specifying its position in the array (its index); in a dictionary
instead, you rather ask for the value associated with a certain key.
So, while arrays are useful to maintain an ordered list of objects,
dictionaries are useful to maintain mappings between certain keys 
and certain values.  In other contexts, dictionaries are called 
hash tables.

\subsection{Creating a NSDictionary}
To create a \texttt{NSDictionary}, you can use the method 
\begin{verbatim}
+dictionaryWithObjectsAndKeys:
\end{verbatim}
which takes as argument a list of objects (to be considered in
couples; the first one is the value, the second is the key); the list
is terminated by \texttt{nil}.  The following example creates the
dictionary used as example in the previous section:
\begin{verbatim}
NSDictionary *dict;

dict = [NSDictionary dictionaryWithObjectsAndKeys:
               @"/opt/picture.png", @"Luca", 
               @"/home/nico/birthday.png", @"Birthday Photo", 
               @"/home/nico/birthday.png", @"Birthday Image", 
               @"/home/marghe/pic.jpg", @"My Sister", nil];
\end{verbatim}
Please note the the keys follow their values rather than preceding 
them.

\subsection{Retrieving a value with objectForKey:}
To retrieve the value associated with a given key, you may 
use the method \texttt{-objectForKey:}, as in the following 
example: 
\begin{verbatim}
NSDictionary *dict;
NSString *path;

dict = [NSDictionary dictionaryWithObjectsAndKeys:
               @"/opt/picture.png", @"Luca", 
               @"/home/nico/birthday.png", @"Birthday Photo", 
               @"/home/nico/birthday.png", @"Birthday Image", 
               @"/home/marghe/pic.jpg", @"My Sister", nil];

// Following will set path to /home/nico/birthday.png
path = [dict objectForKey: @"Birthday Image"]; 
\end{verbatim}
If the key is not in the dictionary, \texttt{-objectForKey:} will return 
\texttt{nil}, for example
\begin{verbatim}
// Following will set path to nil
path = [dict objectForKey: @"My Mother"]; 
\end{verbatim}
(assuming \texttt{dict} is the one created in the previous example).
In real life applications, in most cases you don't know if the key is
in the dictionary or not until you try retrieving the value associated
with it.  In these cases, you need to check the result of
\texttt{objectForKey:} to make sure it's not \texttt{nil} before using 
it.  For example, assuming that \texttt{dict} is a dictionary, you
would normally do --
\begin{verbatim}
NSString *imageName = @"My Father";
NSString *path;

path = [dict objectForKey: imageName];

if (path == nil)
  {
    // This means the dictionary does not contain it
    NSLog (@"Don't know the path to the image '%@'", imageName);
  }
else
  {
    // Do something with path
  }
\end{verbatim}
This is also the standard way to check that a key is contained in a
dictionary -- you call \texttt{objectForKey:} and compare the result
with \texttt{nil}.

\subsection{Enumerating all the Keys and Values}
Sometime, you need to iterate over all the key/value pairs 
in a dictionary.  To do this, you use the method \texttt{-allKeys} 
to retrieve an array of all the keys in the dictionary; this array 
contains all the keys, in no particular (ie random) order. 
You can then cycle over this array, and for each key retrieve 
its value.  The following example prints out all the key-values 
in a dictionary:
\begin{verbatim}
void
describeDictionary (NSDictionary *dict)
{ 
  NSArray *keys;
  int i, count;
  id key, value;

  keys = [dict allKeys];
  count = [keys count];
  for (i = 0; i < count; i++)
  {
    key = [keys objectAtIndex: i];
    value = [dict objectForKey: key];
    NSLog (@"Key: %@ for value: %@", key, value);
  }
}
\end{verbatim}
As usual, this code is just an example of how to enumerate all the 
entries in a dictionary; in real life, to get a description of a NSDictionary, 
you just do \texttt{NSLog (@"\%@", myDictionary);}.

\subsection{NSMutableDictionary}
An \texttt{NSDictionary} is immutable.  If you need a dictionary which
you can change, then you should use \texttt{NSMutableDictionary}.

\texttt{NSMutableDictionary} is a subclass of \texttt{NSDictionary},
so that you can do with an \texttt{NSMutableDictionary} everything
which you can do with a simple \texttt{NSDictionary} plus, you can
modify it.

To set the value of a key in a dictionary, you can use
\texttt{-setObject:forKey:}.  If the key is not yet in the dictionary,
it is added with the given value.  If the key is already in the
dictionary, its previous value is removed from the dictionary (which
has the important side-effect that it is sent a \texttt{-release}
message), and the new one is is put in its place (the new one receives
a \texttt{-retain} message when it is added, as usual).  You should
note that, while values are retained, keys are instead copied.

So, if you use a \texttt{NSMutableDictionary}, you can create 
our usual dictionary as follows: 
\begin{verbatim}
NSMutableDictionary *dict;

dict = [NSMutableDictionary new];
AUTORELEASE (dict);
[dict setObject: @"/opt/picture.png"        
         forKey: @"Luca"]; 
[dict setObject: @"/home/nico/birthday.png" 
         forKey: @"Birthday Photo"];
[dict setObject: @"/home/nico/birthday.png" 
         forKey: @"Birthday Image"];
[dict setObject: @"/home/marghe/pic.jpg"    
         forKey: @"My Sister"];
\end{verbatim}

To remove a key and its associated value, you can just use the method 
\texttt{-removeObjectForKey:}: 
\begin{verbatim}
[dict removeObjectforKey: @"Luca"]; 
\end{verbatim}
this will remove the key \texttt{@"Luca"} and its associated value 
from the mutable dictionary.

\section{Property Lists}
As promised, in this last section we are going to have a quick look at
a very interesting feature of GNUstep: the support for property lists.

As we saw, it is very easy to save/read a string to/from a file.
GNUstep goes well beyond this.  You can save/read any object to/from a
file, using the archiving/dearchiving facilities; these facilities are
great, but save/read objects in a binary format.  For arrays and
dictionaries containing only strings, GNUstep provides other
facilities (the ones we are interested on here) to save/read them from
files in a human-readable (and human-modifiable) format, called a
\emph{property list}.  Actually, the set of objects which you can save
and read in this way is much more general:
\begin{enumerate}
\item Any string can be saved/read as a property list.
\item Any array or dictionary containing only objects which can be
saved/read as a property list can be also saved/read as a property
list.
\end{enumerate}
In practice, any array or dictionary which contains other arrays and
dictionaries can be saved/read provided that in the end all the
'final' objects are strings.

To save or read a dictionary or an array of the correct type, you
write or read them to/from file in the same way as you do with
strings, that is, you use the method \texttt{-writeToFile:atomically:}
to write them to files, and the method
\texttt{+dictionaryWithContentsOfFile:} (for dictionaries) or
the method \texttt{+arrayWithContentsOfFile:} (for arrays) 
to read them from files.

For example, the following code creates a dictionary which 
is used to store, for each person, some info; then it saves 
it to a file in the form of a property list: 
\begin{verbatim}
#include <Foundation/Foundation.h>

int 
main (void)
{
  NSAutoreleasePool *pool = [NSAutoreleasePool new];
  NSMutableDictionary *dict;
  NSDictionary *dict2;

  dict = [NSMutableDictionary new];
  AUTORELEASE (dict);

  dict2 = [NSDictionary dictionaryWithObjectsAndKeys: 
             @"Nicola", @"Name",
             @"Pero", @"Surname",
             @"n.pero@mi.flashnet.it", @"Email", nil];
  [dict setObject: dict2  forKey: @"Nicola"];

  dict2 = [NSDictionary dictionaryWithObjectsAndKeys: 
             @"Ettore", @"Name",
             @"Perazzoli", @"Surname",
             @"ettore@helixcode.com", @"Email", nil];
  [dict setObject: dict2  forKey: @"Ettore"];

  dict2 = [NSDictionary dictionaryWithObjectsAndKeys: 
             @"Richard", @"Name",
             @"Frith-Macdonald", @"Surname",
             @"richard@brainstorm.co.uk", @"Email", nil];
  [dict setObject: dict2  forKey: @"Richard"];

  if ([dict writeToFile: @"/home/nico/testing" atomically: YES])
    {
      NSLog (@"Success");
    }
  else
    {
      NSLog (@"Failure");
    }

  return 0;
}
\end{verbatim}
I tried it on my machine (I encourage you to do the same on yours) 
and it wrote the following into my \texttt{/home/nico/testing} file: 
\begin{verbatim}
{
    Ettore = {
        Email = "ettore@helixcode.com";
        Name = Ettore;
        Surname = Perazzoli;
    };
    Nicola = {
        Email = "n.pero@mi.flashnet.it";
        Name = Nicola;
        Surname = Pero;
    };
    Richard = {
        Email = "richard@brainstorm.co.uk";
        Name = Richard;
        Surname = "Frith-Macdonald";
    };
}
\end{verbatim}
As you see, it is a very simple format, and very nice to read and edit
manually.  The " (speech-marks) are used whenever a string contains
spaces or special characters; they are omitted for simple strings.
Dictionaries are saved as in 
\begin{verbatim}
{
  Email = "richard@brainstorm.co.uk";
  Name = Richard;
  Surname = "Frith-Macdonald";
}
\end{verbatim}
(Note that they use curly brackets, and each key/value pairs is followed 
by a \texttt{;} (semicolon), even the last one).  
Arrays (not shown in the previous example) are saved as in 
\begin{verbatim}
(
 "Nicola Pero", 
 "Ettore Perazzoli", 
 "Richard Frith-Macdonald"
)
\end{verbatim}
(they use round brackets, and each entry is followed by a comma, except 
the last one).
You can find other examples of this format by looking inside 
your \texttt{~/GNUstep} directory.

One of the advantages of this format is that it is very general, but
still very portable and simple.  It is so easy and simple (unless
\texttt{XML}) that you can write a complete parser/unparser for a new
platform in one or two days.

Once you have saved the data to a file in the form of a property list,
you can easily retrieve it from the same file.  The following code
retrieves the data from the file \texttt{/home/nico/testing}, and
prints out the \texttt{Email} of \texttt{Ettore}:
\begin{verbatim}
#include <Foundation/Foundation.h>

int 
main (void)
{
  NSAutoreleasePool *pool = [NSAutoreleasePool new];
  NSDictionary *dict;
  NSDictionary *dict2;
  NSString *email;

  dict = [NSDictionary dictionaryWithContentsOfFile: 
                         @"/home/nico/testing"];
  dict2 = [dict objectForKey: @"Ettore"];
  email = [dict2 objectForKey: @"Email"];

  NSLog (@"Ettore's Email is: %@", email);

  return 0;
}
\end{verbatim}
\end{document}
