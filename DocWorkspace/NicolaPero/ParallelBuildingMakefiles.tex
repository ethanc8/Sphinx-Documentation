%%%
%%% Nicola Pero
%%%
%%%
%%%
%%%
%%%

\documentclass[a4paper]{article}
\begin{document}
\author{Nicola Pero nicola.pero@meta-innovation.com}
\title{Parallel Building with GNUstep Makefiles}
\date{First Edition: March 2010}
\maketitle

\section{Introduction}
In this tutorial we will learn how GNUstep make parallelizes building
your project.  You will learn how to organize your GNUmakefiles to
maximize and control the parallelization of your build.  This is an
advanced topic; please refer to ``Writing GNUstep Makefiles'' and
``More on GNUstep Makefiles'' for an introduction to GNUstep make.

This tutorial only applies to gnustep-make 2.4.0 or later; parallel
building was only partially supported before 2.4.0.

\section{Parallel building}

\subsection{What is parallel building}
By default, gnustep-make builds software in ``serial'' mode.  Each
step of the building process is executed in a well-defined order.  A
step won't start until the previous step has completed.

If you have more than one CPU, or if you have a modern multi-core CPU,
you can speed up your build significantly by telling gnustep-make to
build software in ``parallel'' mode.  In this case, whenever possible
gnustep-make will run a number of tasks in parallel.

There are a number of simple rules that gnustep-make follows to decide
when tasks can be run in parallel, or when there is a sequential order
to be respected.  They are the main topic of this tutorial.

\subsection{Doing a parallel build}
To do a parallel build, you just need to pass the '-j' option to make.
For example,

\begin{verbatim}
make -j
\end{verbatim}

'-j' will try to run as many subprocesses as possible.  You can limit
it by adding a number, for example 'make -j 4' might be a good one.

If you try this out on a project containing many files to compile, and
if you have a machine which can run multiple flows of execution in
parallel, you should get a much faster build as the files are compiled
in parallel.

In the rest of the tutorial we'll learn how gnustep-make decides which
steps are done in parallel and which steps are done in serial order.

\section{Single instance}
In this section we look at building a ``single instance'', that is, a
single thing.  For example, a single tool.  We'll cover building more
than one thing (for example, two tools, or a library and a tool) in
the following sections.

\subsection{Compiling multiple files}
GNUstep make automatically compiles multiple files for the same
instance in parallel.

Let's look at a very simple example: an Objective-C tool composed of
two source files --

\begin{verbatim}
include $(GNUSTEP_MAKEFILES)/common.make

TOOL_NAME = HelloWorld
HelloWorld_OBJC_FILES = HelloWorld.m main.m

include $(GNUSTEP_MAKEFILES)/tool.make
\end{verbatim}

In this case, if you use 'make -j' you will see that the two files are
compiled in parallel.  So, the process is really composed of two
steps:

\begin{verbatim}
Step 1. Compiling HelloWorld.m and main.m (in parallel)
Step 2. Linking them together
\end{verbatim}

\subsection{Adding before-xxx-all and after-xxx-all steps}
If you need to add some custom steps before or after your instance is
built, you can use before-xxx-all and after-xxx-all rules.  These
steps will always be executed exactly before and after the compilation
of your instance.  In other words, the building process for the
HelloWorld tool is really composed of the following steps:
\begin{verbatim}
Step 0. Execute before-HelloWorld-all:: if it exists
Step 1. Compiling HelloWorld.m and main.m (in parallel)
Step 2. Linking them together
Step 3. Execute after-HelloWorld-all:: if it exists
\end{verbatim}

For example, let's say that for some reason you need to create a
header file, HelloWorld.h, which is then included by your Objective-C
files HelloWorld.m and main.m.  Obviously this needs to happen before
any compilation takes place.

To do so, in your GNUmakefile.postamble you would add the following:
\begin{verbatim}
before-HelloWorld-all::
       cp HelloWorld.h.in HelloWorld.h
\end{verbatim}
Please note that there should be a TAB character before the 'cp', and
that if you don't have a GNUmakefile.postamble, you can simply put
this rule in your GNUmakefile, at the end of it, after all the
gnustep-make makefiles have been included.

In the real world, this would probably be a more complicated rule,
creating HelloWorld.h from HelloWorld.h.in by using sed or some other
file editing tool.  But whatever it is, you can simply have your code
executed before any compilation is done by placing it in a
before-HelloWorld-all:: rule.

Obviously if your tool is called HelloMoon, then you would place your
code into a before-HelloMoon-all:: rule.

This code would always be executed serially, before the tool is
compiled.

\section{Multiple instances}

\subsection{Multiple instances of the same type}
Multiple instances of the same type are built in parallel.

Let's say that you are building two tools instead of one.  Your
GNUmakefile might look like the following one:

\begin{verbatim}
include $(GNUSTEP_MAKEFILES)/common.make

TOOL_NAME = HelloWorld HelloMoon
HelloWorld_OBJC_FILES = HelloWorld.m main1.m
HelloMoon_OBJC_FILES = HelloMoon.m main2.m

include $(GNUSTEP_MAKEFILES)/tool.make
\end{verbatim}

When you build this in gnustep-make using 'make -j', the two tools
will be considered independent and be built in parallel.

So, the top-level flow looks as follows:

\begin{verbatim}
Step 0. Execute before-all:: if it exists
Step 1. Build HelloWorld and HelloMoon in parallel
Step 2. Execute after-all:: if it exists
\end{verbatim}

Building each of the HelloWorld and HelloMoon tools would follow the
flow described in the previous section, meaning that the compilation
of the files for each of them will be parallelized too!  If you build
with 'make -j' (and your CPU can make 4 concurrent flows of
execution), for example, it is likely that all the 4 files will be
compiled in parallel.  The two tools are built in parallel, and then
each of them fires the compilation of their 2 files in parallel.  As a
result, all of the 4 files are actually built in parallel.  All the
other stages of building the tools are also done in parallel; for
example, the tools are linked independently, in parallel.

\subsection{Instances of different types}
Instances of different types are not built in parallel; they are built
in the order in which the gnustep-make make fragments (tool.make,
library.make, etc) are included.

This behaviour is backwards-compatible with previous releases of
gnustep-make and it's important to know about it (later on we'll
explain how you do a parallel build in this case by moving the
instances into different subdirectories and using the new
parallel-subdirectories.make makefile).

For example, if your GNUmakefile builds a library and a tool, they
will be always be built separately and in serial order.  If you
include tool.make before library.make, the tool will be built first;
if you include library.make first, the library will be built first.

Let's look at a quick example --

\begin{verbatim}
include $(GNUSTEP_MAKEFILES)/common.make

LIBRARY_NAME = HelloMoon
HelloMoon_OBJC_FILES = HelloMoon.m HelloMars.m

TOOL_NAME = HelloWorld
HelloWorld_OBJC_FILES = HelloWorld.m main1.m

include $(GNUSTEP_MAKEFILES)/library.make
include $(GNUSTEP_MAKEFILES)/tool.make
\end{verbatim}

In this case, the HelloMoon library will be built before the
HelloWorld tool because library.make is included before tool.make.

So, the building in this case will work as follows:

\begin{verbatim}
Step 0. Execute before-all:: if it exists
Step 1. Build HelloMoon
Step 2. Build HelloWorld
Step 3. Execute after-all:: if it exists
\end{verbatim}


\section{Subdirectories}
You may have felt that the rules presented in the previous section
were rather arbitrary.  Instances of the same type are built in
parallel, while instances of different types are not.  What if you
need something different ?

gnustep-make 2.4.0 introduces two new makefile fragments
(serial-subdirectories.make and parallel-subdirectories.make) that
allows you to have total control on the order and the parallelization
in which instances are built.  We will refer to them as the
``subdirectories'' makefiles.  They are meant to replace the
traditional aggregate.make in the long-term.

\subsection{Examples}
Let's say that you want to build a tool and a library.  To use the new
subdirectories makefiles, first of all you need to put the tool and
the library into two separate subdirectories.  For example, you could
put the tool into a subdirectory called Tools, and the library into a
subdirectory called Source (they could be called anything you want).
Each of them will have its own standalone GNUmakefile.

At the top-level, you create a GNUmakefile that will drive the
building of the subdirectories.  Here is an example:

\begin{verbatim}
include $(GNUSTEP_MAKEFILES)/common.make

SERIAL_SUBDIRECTORIES = Source Tools

include $(GNUSTEP_MAKEFILES)/serial-subdirectories.make
\end{verbatim}

This will cause gnustep-make to go into the SERIAL\_SUBDIRECTORIES in
order, and build them serially, one by one.  So, it will first build
Source (that contains our library) and then Tools (that contains our
tool).

If you want the two subdirectories to be built in parallel, you just
need to use parallel-subdirectories.make, as follows:

\begin{verbatim}
include $(GNUSTEP_MAKEFILES)/common.make

PARALLEL_SUBDIRECTORIES = Source Tools

include $(GNUSTEP_MAKEFILES)/parallel-subdirectories.make
\end{verbatim}

This will tell gnustep-make that it should fork off two subprocesses,
and build the Source and Tools subdirectories in parallel.

\subsection{Advanced usage of subdirectories}
If you now nest subdirectories at different levels, you can control
exactly what you want to be built in parallel and what you want to be
built in serial order.

For example, you may have a few frameworks that can be built in
parallel, followed by a few tools that can be built in parallel, but
the frameworks always need to be built before the tools.

In this case, you could have a Frameworks/ and Tools/ subdirectories,
built using serial-subdirectories.make so that they are built in that
order.  Inside Frameworks/ you then have a GNUmakefile that uses
parallel-subdirectories.make to build a number of frameworks in
parallel (each of them in its own subdirectory), and similarly in
Tools/ to build the tools in parallel.

The advantage of parallelizing the build of entire directories might
not be apparent simply because your CPU might not support enough
concurrency to do more than a few things in parallel.  But if you
simply organize your directory structure and use
parallel-subdirectories.make and serial-subdirectories.make in a way
that makes sense, you'll get the full benefit of more parallel-capable
CPUs as soon as you get them.  Your build will scale really well when
you add CPUs and cores.

Moreover, when you parallelize building entire directories you not
only parallelize the compilation steps, but all the other ones;
linking and even before-all:: and after-all:: rules.

For example, if you have in one directory a before-all:: rule which
takes 10 seconds to run, since it is executed serially it may prevent
anything else from running for these 10 seconds.  For 10 seconds all
your other CPUs or cores might be sitting idle.  But if you
parallelize the build of that directory with the build of another
directory, even during these 10 seconds, the other CPUs or cores might
still be busy building things in the other directory.

\subsection{Backwards compatibility between subdirectories and aggregate.make}
If you need your software to build with gnustep-make < 2.4.0, then you
can't really use parallel-subdirectories.make or
serial-subdirectories.make because they weren't available in older
versions of gnustep-make.  You need to use aggregate.make.

aggregate.make will always do a serial build in older versions of
gnustep-make.  In newer versions, it will do a serial build by
default, but will do a parallel build if you set 

\begin{verbatim}
GNUSTEP_USE_PARALLEL_AGGREGATE = yes
\end{verbatim}

So, here is how to build a set of subdirectories in serial order --

\begin{verbatim}
include $(GNUSTEP_MAKEFILES)/common.make

SUBPROJECTS = Source Tools

include $(GNUSTEP_MAKEFILES)/aggregate.make
\end{verbatim}

and here is how to build them in parallel oder --

\begin{verbatim}
include $(GNUSTEP_MAKEFILES)/common.make

SUBPROJECTS = Source Tools

GNUSTEP_USE_PARALLEL_AGGREGATE = yes
include $(GNUSTEP_MAKEFILES)/aggregate.make
\end{verbatim}

This last example will simply do a parallel build on newer
gnustep-makes, and a serial build on older ones that don't support it.

aggregate.make will be slowly phased out in the following years, so if
you don't need to support older versions of gnustep-make, you could
simply use serial-subdirectories.make and
parallel-subdirectories.make.

\section{Subprojects}
\subsection{Why do subprojects exist ?}
Before version 2.4.0, gnustep-make did not support having source files
in subdirectories.  For example, you couldn't write:

\begin{verbatim}
include $(GNUSTEP_MAKEFILES)/common.make

TOOL_NAME = HelloWorld
HelloWorld_OBJC_FILES = Subdirectory/HelloWorld.m

include $(GNUSTEP_MAKEFILES)/tool.make
\end{verbatim}

This works if you use gnustep-make 2.4.0 (or later); but wouldn't work
with previous versions of gnustep-make, because HelloWorld.m is in a
subdirectory.  So, people who wanted to organize their source files in
different subdirectories had to use ``subprojects''.  You created a
GNUmakefile in each of the subdirectories (which would build using
subproject.make), then specifies that the tool had some subprojects.
So, the tool's GNUmakefile would have been --

\begin{verbatim}
include $(GNUSTEP_MAKEFILES)/common.make

TOOL_NAME = HelloWorld
HelloWorld_SUBPROJECTS = Subdirectory

include $(GNUSTEP_MAKEFILES)/tool.make
\end{verbatim}

and in the subdirectory you'd have had a GNUmakefile such as the
following one --

\begin{verbatim}
include $(GNUSTEP_MAKEFILES)/common.make

SUBPROJECT_NAME = Subdirectory
Subdirectory_OBJC_FILES = HelloWorld.m

include $(GNUSTEP_MAKEFILES)/subproject.make
\end{verbatim}

You'll appreciate how easier it is to specify source files in
subdirectories directly in the tool's GNUmakefile (as allowed by
gnustep-make 2.4.0), without having to have an additional GNUmakefile
in each subdirectory.

\subsection{Most of the times, you do not need subprojects any more}
As explained, in recent versions of gnustep-make, there is no longer
any need for subprojects just to organize your source files in
subdirectories.  You can simply put your source files into
subdirectories, then list them all in your GNUmakefile.  The project
is simpler, and gnustep-make is able to parallelize as much as
possible since it will (potentially) be able to compile all the files
in parallel.

\subsection{Subprojects and paralell building}
If you ever need to use subprojects in a parallel build, you need to
know that, for backwards-compatibility, subprojects are built in the
order they are specified, and are built before the rest of the source
files are compiled.

So, in the example above of a tool HelloWorld with a Subdirectory
subproject, the build flow to build the tool is as follows:

\begin{verbatim}
Step 0. Execute before-HelloWorld-all:: if it exists
Step 1. Build Subdirectory
Step 2. Build HelloWorld
Step 3. Execute after-HelloWorld-all:: if it exists
\end{verbatim}

if you had more than one subproject, they would be built in the
specified order.  Most likely, they are independent though, and should
be built in parallel.  You can very simply obtain that result by
getting rid of subprojects altogether and listing all files in the
tool's GNUmakefile.  This will also get rid of the intermediate
linking steps for subprojects, speeding up your build even more.

\section{Disabling parallel building}
If parallel building is causing problems, you can always turn it off.
There are three ways of turning it off:

\begin{enumerate}
\item globally, by configuring gnustep-make with ./configure
  --disable-parallel-building.  This sounds a bit extreme, but it's an
  option if you want everything to be always built serially.
\item for a specific GNUmakefile, by adding
\begin{verbatim}
GNUSTEP_MAKE_PARALLEL_BUILDING = no
\end{verbatim}
to the GNUmakefile, just before including common.make.  This is
probably the most useful trick; if for any reason you find a
GNUmakefile is not working with parallel building and you don't know
how to fix it, you can disable parallel building in that GNUmakefile
while still leaving it on for everything else.
\item for a specific compilation run, by simply not using '-j' when
invoking make. :-)
\end{enumerate}

\end{document}
